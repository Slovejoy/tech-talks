\section{Motivation}
\frame{\sectionpage}

\begin{frame}{Threat Models using Hardware Vulnerabilities}
    Academic researchers have demonstrated CPU vulnerabilities for decades, but the Spectre+Meltdown publications in Jan 2018 introduced them into modern threat models. They can be used in software exploit chains in the following spots,
    \begin{block}{Places in Existing Exploit Chains for CPU Vulnerabilities}
        \begin{itemize}
            \item Local Privilege Escalation (\$250k+)
            \item Browser Sandbox Escapes (\$250k)
            \item Virtualization Escape (\$200k)
            \item Bypasses for security protections like \href{https://arxiv.org/abs/2406.08719}{\color{pink}ARM MTE+PAC}, ASLR, and Intel CET
            \begin{itemize}
                \item Most microarchitectural vulnerabilities tend to yield arbitrary reads across the system.
            \end{itemize}
            \item Pivoting into coprocessors, namely TEEs, but also basebands and DSPs.
       \end{itemize}
    \end{block}
    Even once discovered, it takes around 1 year to patch hardware flaws if they are patched at all. 
    \note{
        \begin{itemize}
            \item Finding hardware bugs is really hard, a few have been found with fuzzing, some by accident, but the majority academics kinda just "conjecture" as try stuff. Sometimes bugs can be gleaned from documentation.
            \item Threat modeling hardware bugs is kinda messed up; in fact, most SoC vendors don't care about instruction-level granularity of code (i.e. they don't care if other folks on the same system know what code you're executing)!
            \item SoC vendors do care that secrets (i.e. dcache) stay protected, such as tokens, keys, and cookies.
            \item See also, the relatively unexplored fingerprinting process.
        \end{itemize}
    }
\end{frame}

\begin{frame}{APT Orchestration of Hardware Vulnerabilities}
    \begin{block}{CVE-2023-38606}
        \begin{itemize}
            \item Exposed GPU MMIO debug registers can disable page protections in the iOS kernel.
            \item The only known hardware vulnerability known ever exploited according to CISA.
            \item Used in \href{https://securelist.com/operation-triangulation-the-last-hardware-mystery/111669/}{\color{pink}{Operation Triangulation Sensation}} against Kaspersky Labs (believed to be USG).
        \end{itemize}
    \end{block}
    Other use cases:
    \begin{itemize}
        \item Stephen Röttger of Google demonstrates that microarchitectural vulnerabilities are useful for escaping the \href{https://googleprojectzero.blogspot.com/2020/02/escaping-chrome-sandbox-with-ridl.html}{\color{pink}{Chrome Sandbox}}.
        \item Carlo Meijer of Midnight Blue uses a \href{https://www.usenix.org/system/files/usenixsecurity23-meijer.pdf}{\color{pink}cache-timing side-channel} to break into the TEE of a Texas Instruments OMAP-L138 SoC. 
    \end{itemize}
    As advanced memory security features become more common, the security landscape will greater reward logical bugs including hardware and microarchitectural flaws. 
    \note{
        \begin{itemize}
            \item It's incredibly hard to detect the exploitation of hardware flaws, there may be some but we just don't know.
            \item In some examples, hardware flaws may be helpful for reverse engineering and introspection as well, not unlike fault injection.
            \item There are tons of papers getting published right now in this field, but not all of these vulnerabilities are practical for exploitation.
        \end{itemize}
    }
\end{frame}